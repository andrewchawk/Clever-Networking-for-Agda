\documentclass{report}

% The coloring distracts the author.
\usepackage[bw]{agda}

\title{\AgdaModule{Network}: An Extensible and Expressive but Slightly Clever Agda Module for the Representation of Various Entities in Computer Networking}

\begin{document}
\maketitle{}

\begin{abstract}
The author presents and explains \AgdaModule{Network}, which is an Agda representation of network packets and connections.  Notable properties of \AgdaModule{Network} include extensibility, expressive typing for protocols and packets, real-world examples, and at least some degree of formal verification.  However, \AgdaModule{Network} \emph{cannot} be directly used to actually establish network connections in the real world; \AgdaModule{Network} is a \texttt{--safe} module and, therefore, does not refer to \AgdaDatatype{IO}, which would be necessary for such establishment.
\end{abstract}

\chapter{Boilerplate Stuff}
This section contains the declaration of the name of this module and some import statements.  These things would not fit too well in any other part of the document.

\section{Options}
First and foremost --- actually, at this point in time, this option is the only one --- this document is safe, i.e., does not make use of postulates or any such nonsense.

\begin{code}
{-# OPTIONS --safe #-}
\end{code}

\section{Module Declaration}
The Agda parts of this paper constitute an Agda module.  Some submodules also exist.

\begin{code}
module Network where
\end{code}

\section{Imported Packages}
The author wishes to \emph{not} rewrite \emph{all} utilities.

\begin{code}
import Data.String
import Data.Maybe.Relation.Unary.Any as DataMaybe using (Any)

open import Function
open import Relation.Binary.HeterogeneousEquality
open import Data.Fin using (Fin; toℕ)
open import Data.String using (String)
open import Data.Nat
open import Data.List
open import Data.Unit.Polymorphic
open import Data.Maybe
open import Data.Bool
open import Data.Product
open import Data.List.Relation.Unary.All using (All)
open import Data.List.Relation.Unary.Any using (Any)
open import Level using (Level)
\end{code}

\part{The Core Types}
To prevent name clashes, this part constitutes an Agda module.

\begin{code}
module Core where
\end{code}

\chapter{Representation of Protocols}
Formally, where \(P\) is a certain subset of networking protocols, \(P\) is characterized by being such that for all elements \(p\) of \(P\), \(p\) can be represented as the combination of a type for network addresses for \(p\), a type for network ports for \(p\), although \(p\) might not explicitly use network ports, and a type for \(p\) packets.  Additionally, any such \(p\) probably has a name, and the name might even be an abbreviation, which could be convenient.

\AgdaRecord{Protocol} is an extension of this representation; where \AgdaBound{p} is some \AgdaRecord{Protocol} \AgdaBound{l} value, the following statements hold:
\begin{itemize}
	\item \AgdaField{Protocol.name} \AgdaBound{p} is a long name of the \AgdaBound{p} protocol.
	\item \AgdaField{Protocol.shortName} \AgdaBound{p} is, optionally, an abbreviation of the long name of the \AgdaBound{p} protocol.
	\item If applicable, \AgdaField{Protocol.addressType} \AgdaBound{p} contains the type of network addresses which are used by the \AgdaBound{p} protocol.
	\item \AgdaField{Protocol.packetType} \AgdaBound{p} is the type of all packets in the \AgdaBound{p} protocol.
	\item \AgdaField{Protocol.isSerialized} \AgdaBound{p} is \AgdaInductiveConstructor{true} if and only if all \AgdaBound{p} packets have serialized equivalents.
	\item If \AgdaBound{p} packets can be encapsulated in packets of any other protocol, then \AgdaField{Protocol.carrierProtocols} \AgdaBound{p} is \AgdaInductiveConstructor{nothing}; otherwise, \AgdaField{Protocol.carrierProtocols} \AgdaBound{p} is a list of protocols which support the encapsulation of \AgdaBound{p} packets.
	\item \AgdaField{Protocol.standaloneHasAddress} \AgdaBound{p} proves that if no protocols are approved for the carrying of the \AgdaBound{p} protocol, then the \AgdaBound{p} protocol has some sort of native addressing scheme.
	\item \AgdaField{Protocol.serializedLikeCarrier} \emph{might} be mostly self-explanatory but indicates that protocols which are carried by serialized protocols must \emph{also} be serialized.

\begin{code}
  record Protocol (l : Level) : Set (Level.suc l) where
    inductive
    field
      name : String
      shortName : Maybe String
      addressType : Maybe Set
      packetType : Set l
      isSerialized : Bool
      carrierProtocols : Maybe (List (Protocol l))
      standaloneHasAddress : Is-just carrierProtocols -> Is-just addressType
      serializedLikeCarrier :  (justProof : Is-just carrierProtocols)
                            -- The following hole should be equivalent to
                            -- Protocol.isSerialized cp, but
                            -- Protocol.isSerialized
                            -- does not work here.  The author is currently
                            -- unaware of a good solution.
                            -> Any (\ cp -> (Bool ∋ {!!}) ≅ true)
                                   (to-witness justProof)
                            -> isSerialized ≅ true
\end{code}

\section{Can \AgdaRecord{Protocol} Unambiguously Represent \emph{Any} Network Protocol?}
\AgdaRecord{Protcol} is good for representing many sorts of protocols but may be incapable of unambiguously representing all imaginable protocols; the author is uncertain of a formal definition of ``network protocol'', and awareness of such a definition would facilitate defining a thing which resembles \AgdaRecord{Protocol} but an really be used to represent \emph{any} network protocol.  The author \emph{may} conduct some more research into the idea of network protocols but, in the meantime, thinks that \AgdaRecord{Protocol} should suffice for most purposes.

\chapter{Representation of Packets}
Without a representation of trasmissions, what is the usefulness of a system for representing protcols?  The author does not care and will define a datatype which facilitates representing the packets of any given protocol.

A packet \(p\) for a protocol \(P\) can be thought of as being a combination of the following attributes:
\begin{itemize}
	\item the \(P\) address of the sender of \(p\),
	\item the \(P\) address of the destination for \(p\), and
	\item the actual content or payload of \(p\).
\end{itemize}

The address type and content type are protocol-specific.  Fortunately, with dependent types, defining such a type is a simple process.  This paper's definition of an abstract packet is as follows:

\begin{code}
  record Packet {l : Level} (protocol : Protocol l) : Set l where
    field
      source
       destination : maybe (\ n -> n) ⊤ (Protocol.addressType protocol)
      content : Protocol.packetType protocol
      serialized : Maybe (Σ ℕ Fin)
      serializationDemand  : Protocol.isSerialized protocol ≅ true
                          -> Is-just serialized
\end{code}

For a given \AgdaRecord{Packet} \AgdaBound{P} value \AgdaBound{p}, the following statements hold:
\begin{itemize}
	\item \AgdaField{Packet.source} \AgdaBound{p} is the address of the sender of \AgdaBound{p}.
	\item \AgdaField{Packet.destination} \AgdaBound{p} is the address of the recipient of \AgdaBound{p}.
	\item \AgdaField{Packet.content} \AgdaBound{p} is the payload of \AgdaBound{p}.
	\item \AgdaField{Packet.serialized} \AgdaBound{p} is, somewhat optionally, a serialized version of packet.
	\item \AgdaField{Packet.serializationDemand} \AgdaBound{p} indicates that if \AgdaBound{p} is a protocol whose packets \emph{must} be serializable, then a serialization of \AgdaBound{p} is immediately available.
\end{itemize}

\part{Specific Protocols}

\chapter{IPv4}
For the sake of brevity and general readability, within this example, the ``\AgdaModule{IPv4}'' prefix is omitted from the names of functions, constants, and such.

\begin{code}
module IPv4 where
\end{code}

RFC 791 describes version 4 of the Internet Protocol, which is also known as ``IPv4'', indicates that IPv4 addresses are 32-bit integers, and defines the IPv4 packet structure.  Specifically, RFC 791 indicates that any IPv4 packet \(p\) consists of the concatenation of the following fields:
\begin{enumerate}
	\item a four-bit version number,
	\item a four-bit definition of the length of the header which \emph{must} be greater than or equal to five,
	\item a six-bit differentiated services code point,
	\item a two-bit explicit congestion notification,
	\item a sixteen-bit integer which defines the total length of \(p\),
	\item a sixteen-bit identification field,
	\item a zero bit,
	\item a bit which indicates whether or not \(p\) must not be fragmented, with a one indicating that the packet must \emph{not} be fragmented},
	\item a bit which indicates whether or not \(p\) is the last of a series of fragmented packets,
	\item a thirteen-bit fragment offset field,
	\item an eight-bit time-to-live field,
	\item an eight-bit value which identifies the payload protocol,
	\item a sixteen-bit checksum of the header,
	\item a thirty-two-bit IPv4 address, indicating the source of \(p\),
	\item a thirty-two-bit IPv4 address, indicating the destination of \(p\),
	\item an options field whose length is calculable, and
	\item a payload of calculable length.
\end{enumerate}

A naive approach involves using \AgdaDatatype{Fin} for everything.  However, the author uses the term ``naive'' because the author prefers the alternative, which involves the use of more expressive datatypes which are specifically designed \emph{for} IPv4, although the \AgdaDatatype{Fin} approach really does work well for some fields.

\section{Addresses}
That at least two fields can be absent from the \AgdaField{Protocol.packetType} type may be immediately obvious; \AgdaRecord{Protocol} has native support for address schemes.  At this point, an IPv4 address type should probably be created, so the author \emph{has} gone and created such a type!  Specifically, the type is \AgdaFunction{Address}, which is defined as follows:

\begin{code}
  Address : Set
  Address = Fin (2 ^ 32)
\end{code}

The definition of \AgdaFunction{Address} follows pretty directly from the RFC's definition of IPv4 addresses; \AgdaDatatype{Fin} \AgdaSymbol(\AgdaNumber{2} \AgdaOperator{\AgdaFunction{^}} \AgdaBound{x}\AgdaSymbol) is the type of the \AgdaBound{x}-bit natural numbers.

\section{The Header Length Field}
Naively, one can say that the header length field is the combination of an appropriate \AgdaDatatype{Fin} number \(n\) and a proof which indicates that \(n \geq 5\).  In this case, the author actually \emph{likes} the naive approach.

\begin{code}
  IHL : Set
  IHL = Σ (Fin (2 ^ 4)) (\ n -> toℕ n ≥ 5)
\end{code}

\section{The Three Option Bits}
An approach to representing the three option bits which yields a not-particularly-readable result involves the use of \AgdaDatatype{Fin} \AgdaSymbol(\AgdaNumber{2} \AgdaOperator{\AgdaFunction{\circumflex}} \AgdaNumber{3}\AgdaSymbol).  However, the option bits an instead be represented as dedicated \AgdaDatatype{Bool} fields in a record type; this approach offers significantly more readability and prevents confusing the purposes of the individual option bits.

\subsection{IPv6-Specific Values}
Some \AgdaDatatype{PayloadProtocol} contructors, e.g., \AgdaInductiveConstructor{Ethernet}, are actually specific to IPv6.  However, IPv6 and IPv4 use the same IP protocol numbers, and the author does not believe that listing all such protocol numbers is in any real way problematic.

\section{The Packet Record}
First and foremost, the packet type references the protocol record, but the protocol record must also reference the packet type.  As such, both definitions are contained within an \AgdaKeyword{interleaved} \AgdaKeyword{mutual} block.

\begin{code}
  interleaved mutual
\end{code}

Moving on, armed with the preceding information which actually pertains to IPv4, a type \AgdaRecord{Packet} can be decently easily defined such that for all \AgdaBound{p} of type \AgdaRecord{Packet}, the following statements hold:
\begin{itemize}
	\item \AgdaField{Packet.versionNumber} \AgdaBound{p} is the version number for the \AgdaField{p} packet.
	\item \AgdaField{Packet.headerLength} \AgdaBound{p} is the combination of a four-bit number \(l\), which is the length of the header of the \AgdaBound{p} packet, and a value which guarantees that \(l \geq 5\).
	\item \AgdaField{Packet.differentiatedServices} \AgdaBound{p} is the differentiated services code point for the \AgdaBound{p} packet.
	\item \AgdaField{Packet.congestionNotification} \AgdaBound{p} is the explicit congestion notification for the \AgdaBound{p} packet.
	\item \AgdaField{Packet.totalLength} \AgdaBound{p} is the total length of the \AgdaBound{p} packet.
	\item \AgdaField{Packet.identification} \AgdaBound{p} is the raw content of the identification field of the \AgdaBound{p} packet.
	\item \AgdaField{Packet.firstFlagBit} \AgdaBound{p} is \emph{reserved} and should be set to \AgdaInductiveConstructor{Fin.zero}.
	\item \AgdaField{Packet.dontFragment} \AgdaBound{p} is true if and only if the \AgdaBound{p} packet \emph{must not} be fragmented into additional packets.
	\item \AgdaField{Packet.moreFragments} \AgdaBound{p} is false if and only if the \AgdaBound{p} packet is the last of a series of fragmented packets or is a standalone packet.
	\item \AgdaField{Packet.fragmentOffset} \AgdaBound{p}
	\item \AgdaField{Packet.timeToLive} \AgdaBound{p}
	\item \AgdaField{Packet.headerChecksum} \AgdaBound{p} is the checksum of the header of the \AgdaBound{p} packet.
	\item \AgdaField{Packet.options} \AgdaBound{p} is the raw options field for the \AgdaBound{p} packet.
	\item \AgdaField{Packet.protocol} \AgdaBound{p} is the protocol of \AgdaField{Packet.payload} \AgdaBound{p}.
	\item \AgdaField{Packet.payload} \AgdaBound{p} is the payload of the \AgdaBound{p} packet.
	\item \AgdaField{Packet.protocolIsUsable} \AgdaBound{p} indicates that the \AgdaField{Packet.protocol} \AgdaBound{p} protocol actually supports having packets encapsulated within IPv4 packets.
	\item \AgdaField{Packet.payloadLengthMatches} ensures that the payload length is actually the payload length which is calculated using \AgdaField{Packet.totalLength} \AgdaBound{p} and \AgdaField{Packet.headerLength p}.
\end{itemize}

\begin{code}
    record Packet (l : Level) : Set (Level.suc l) where
      field
        versionNumber : Fin (2 ^ 4)
        headerLength : IHL
        differentiatedServices : Fin (2 ^ 6)
        congestionNotification : Fin (2 ^ 2)
        totalLength : Fin (2 ^ 16)
        identification : Fin (2 ^ 16)
        firstOptionBit : Bool
        dontFragment : Bool
        moreFragments : Bool
        fragmentOffset : Fin (2 ^ 13)
        timeToLive : Fin (2 ^ 8)
        headerChecksum : Fin (2 ^ 16)
        options : Fin (2 ^ (toℕ (proj₁ headerLength) ∸ 5))
        protocol : Core.Protocol l
        payload : Core.Packet protocol
        protocolIsUsable : maybe (Any (_≅ protocol)) ⊤ (Core.Protocol.carrierProtocols protocol)
        payloadLengthMatches :
          let expectedLength = 2 ^ toℕ totalLength ∸ toℕ (proj₁ headerLength) in
          let actualLength = Data.Maybe.map proj₁ (Core.Packet.serialized payload) in
          actualLength ≅ just expectedLength
\end{code}

\subsection{On \AgdaField{Packet.payload}}
The current typing scheme does \emph{not} ensure that for any appropriate \AgdaBound{p}, \AgdaField{Packet.protocol} \AgdaBound{p} actually describes the structure of \AgdaField{Packet.payload} \AgdaBound{p}; the \AgdaField{Packet.protocol} type is just a dumb \AgdaDatatype{Fin} type.  However, as indicated by the use of ``current'', this bit of information may eventually become outdated; such indication is strictly intentional.

\section{The Protocol Record}
With this information, the IPv4 protocol can be considered to have the following characteristics:
\begin{itemize}
	\item The long name of IPv4 is ``Internet Protocol version 4''.
	\item IPv4 has a short name.  This short name is, obviously, ``IPv4''.
	\item IPv4 addresses are 32-bit numbers, which are represented by \AgdaFunction{Address}.
\end{itemize}

Accordingly, the IPv4 protocol can be defined with an \AgdaRecord{Core.Protocol} record as follows:

\begin{code}
    protocol : (l : Level) -> Core.Protocol (Level.suc l)
    protocol l = record
      {name = "Internet Protocol version 4"
      ;shortName = just "IPv4"
      ;addressType = just Address
      ;packetType = Packet l
      ;isSerialized = true
      ;carrierProtocols = {!!}
      ;serializedLikeCarrier = \ _ _ -> refl
      ;standaloneHasAddress = \ _ -> DataMaybe.Any.just _
      }
\end{code}

\section{The Addressful Packet Type}
Really, because \AgdaRecord{Core.Packet} exists, no explicit definition of a type for addressful IPv4 packets is necessary.  For any appropriate \AgdaBound{p}, \AgdaRecord{Core.Packet} \AgdaBound{p} is the type of packets which adhere to the \AgdaBound{p} protocol, and \AgdaFunction{protocol} is a description of the IPv4 protocol; therefore, \AgdaRecord{Core.Packet} \AgdaFunction{protocol} is the type of IPv4 packets.
\end{document}
